% vim:ft=tex:
%
\documentclass[UTF8]{ctexart}
\usepackage{amsmath}
\usepackage{listings}

\lstset{
    basicstyle          =   \sffamily,          % 基本代码风格
    keywordstyle        =   \bfseries,          % 关键字风格
    commentstyle        =   \rmfamily\itshape,  % 注释的风格,斜体
    stringstyle         =   \ttfamily,  % 字符串风格
    flexiblecolumns,                % 别问为什么,加上这个
    numbers             =   left,   % 行号的位置在左边
    showspaces          =   false,  % 是否显示空格,显示了有点乱,所以不现实了
    numberstyle         =   \zihao{-5}\ttfamily,    % 行号的样式,小五号,tt等宽字体
    showstringspaces    =   false,
    captionpos          =   t,      % 这段代码的名字所呈现的位置,t指的是top上面
    frame               =   lrtb,   % 显示边框
}

\lstdefinestyle{Python}{
    language        =   Python, % 语言选Python
    basicstyle      =   \zihao{-5}\ttfamily,
    numberstyle     =   \zihao{-5}\ttfamily,
    keywordstyle    =   \color{blue},
    keywordstyle    =   [2] \color{teal},
    stringstyle     =   \color{magenta},
    commentstyle    =   \color{red}\ttfamily,
    breaklines      =   true,   % 自动换行,建议不要写太长的行
    columns         =   fixed,  % 如果不加这一句,字间距就不固定,很丑,必须加
    basewidth       =   0.5em,
}
\title{
	人工智能基础作业1
}
\author{
	林乐天 --- \texttt{2300012154@stu.pku.edu.cn}
}

\begin{document}
\maketitle
\section{问答1}
\subsection{问题}请简述什么是贝叶斯定理,什么是最大似然估计(MLE),什么是最大后验估计
(MAP)。
\subsection{答案}

\paragraph{贝叶斯定理}贝叶斯定理是一个描述对事件发生可能性的信心的公式,基于对已有可能与事件相关的先验知识来描述时间发生的可能性。其公式为:$P(A|B)=\frac{P(B|A)P(A)}{P(B)}$,称$P(B|A)$先验概率,称$P(A|B)$为后验概率。

\paragraph{最大似然估计}最大似然估计是一种优化方法,是在假定数据随机无偏的情况下,通过使得$P(B|A)$最大的方法,使得模型最大限度的拟合现实情况,对数据集的要求很高。

\paragraph{最大后验估计}最大后验估计也是一种优化方法,本质上是不信任数据集能真实反映现实情况,通过假设一个先验分布,最大化$P(A|B)$即最大化$P(B|A)*P(A)$来优化模型的方法,对先验分布比较敏感。

\section{问答2}
\subsection{问题}
设$X~N(\mu, \sigma^2)$, $\mu,\sigma^2$为未知参数,$x_1\dots x_n$是来自$X$的样本值,求$\mu,\sigma^2$的最大似然估计量。
\subsection{答案}$$\mu = \frac{1}{n}\sum_i={1}^n x_n$$
$$\sigma^2 = \frac{1}{n}\sum_{i=1}^n(x_n - \mu)^2$$

\section{问答3}
\subsection{问题}
请简述分类问题与回归问题的主要区别。
\subsection{答案}
回归问题输出的是一个连续值,是对一个连续函数的预测和拟合,其标签是连续的;而分类问题输出的是一个离散值,是对一个离散值的预测,其标签是离散的。
\section{问答4}
\subsection{问题}
请简述有监督学习与无监督学习的主要区别。
\subsection{答案}
有监督学习除了给出数据,还给出了lable;无监督学习只给出数据,不给出label。

\section{问答5}
\subsection{问题}给定数据$D = /{(x_1, y_1), (x_2, y_2), … , (x_n, y_n)/}$,
用一个线性模型估计最接近真实$y$(ground truth )的连续标量 $Y$,$f(x_i ) = Wx_i + b$, such that $f(x) \approx y$.
求最优 $(w^*, b^*)$使得$f(x)$与$y$之间的均方误差最小
并解释$(w^*, b^*)$何时有 closed form 解,何时没有 closed form 解。
\subsection{答案}
先写出均方误差:$$L(W,b) = \frac{1}{n}\sum{Wx_i+b-y_i}$$

$$L(\beta)=\frac{1}{n}(A\beta-Y)^T(A\beta-Y)$$其中,$A$包含了数据的信息,且加了一行1;$\beta$是前文$W$和$b$的组合。

对$\beta$求偏导并求零点,化简得:$(A^TA)\beta = A^TY$

如果$(A^TA)$可逆,则其有closed form的解,否则没有。


\section{问答6}
\subsection{问题}Ridge regression 问题的解具有什么特点,为什么?Lasso 问题的解具有什么特点?为什
么?
\subsection{答案}
\paragraph{Ridge regression}
这种方法的解相对来说其绝对值较小。由于其为$L_2$距离,对绝对值大小比较敏感,且容易在非坐标轴的位置和解域相交,所以绝对值小。
\paragraph{Lasso regression}
这种方法的解相对维度较小,会出现大量的0。由于其为$L_1$距离,解域很容易在坐标轴上相交,于是如此。
\section{问答7}
\subsection{问题}
请从 model function、loss function、optimization solution 三个方面比较 Linear regression
与 Logistic regression 的异同。
\subsection{答案}
\paragraph{model function}
对于Linear来说,model function只是一个单纯的对X的linear conbination,在加上一个bias,是一个纯粹的线性形式$y=Wx+b$

而对于Logistic regression来说,为了让结果可以解释,要在线性组合外面套一个sigmoid函数,就使得整个函数并非完全的线性,输出值也被控制在了[0,1]上。

\paragraph{loss function}
对于Linear来说,更常用的是基本的Square Loss,由于取值范围相对较大,使用均方误差收敛速度尚可,故如此。

而对于Logistic来说,若使用Square Loss,会在结果严重偏离正确结果的时候不能收敛,其梯度会变成0,于是常用Cross Ertnopy Loss,从而保证其在梯度下降的时候可以正常运作。

\paragraph{optimization solution}
二者在这个方面具有一致性。两个函数都是用梯度下降的方法来进行优化的,所以二者在这个方面是相同的。


\section{问答8}
\subsection{问题}
K-近邻分类器的超参数是什么?怎么选择 K-近邻分类器的超参数?
\subsection{答案}
\paragraph{超参数}其超参数有两个,一个是K的值,另一个是距离量度的选择。
\paragraph{选择}通过在测试集上进行调试找到合适的超参数,在数据集不够大的时候可以进行交叉验证。
\end{document}
